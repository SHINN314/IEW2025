\documentclass[uplatex,dvipdfmx,a4paper,10pt]{jsarticle}
\usepackage{graphicx}
\usepackage{amsmath}
\usepackage{latexsym}
\usepackage{multirow}
\usepackage{url}
\usepackage[separate-uncertainty]{siunitx}
\usepackage{physics}
\usepackage{enumerate}
\usepackage{bm}
\usepackage{pdfpages}
\usepackage{pxchfon}
\usepackage{tikz}
\usepackage{float}
\usepackage{listings}
\usepackage{amssymb}

% lstlistingのsetting
\lstset{
    basicstyle={\ttfamily},
    identifierstyle={\small},
    commentstyle={\smallitshape},
    keywordstyle={\small\bfseries},
    ndkeywordstyle={\small},
    stringstyle={\small\ttfamily},
    frame={tb},
    breaklines=true,
    columns=[l]{fullflexible},
    numbers=left,
    xrightmargin=0zw,
    xleftmargin=3zw,
    numberstyle={\scriptsize},
    stepnumber=1,
    numbersep=1zw,
    lineskip=-0.5ex
}

% tikz setting
\usepackage{tikz}
\usetikzlibrary{automata, intersections, calc, arrows, positioning, arrows.meta}

% theories setting (for japanese language)
\usepackage{amsmath}
\usepackage{amsthm}

\theoremstyle{definition}
\newtheorem{thm}{定理}[section]
\newtheorem{lem}[thm]{補題}
\newtheorem{prop}[thm]{命題}
\newtheorem{cor}[thm]{系}
\newtheorem{ass}[thm]{仮定}
\newtheorem{conj}[thm]{予想}
\newtheorem{dfn}[thm]{定義}
\newtheorem{rem}[thm]{注}

\newtheorem*{thm*}{定理}
\newtheorem*{lem*}{補題}
\newtheorem*{prop*}{命題}
\newtheorem*{cor*}{系}
\newtheorem*{ass*}{仮定}
\newtheorem*{conj*}{予想}
\newtheorem*{dfn*}{定義}
\newtheorem*{rem*}{注}

% \renewcommand{\rmdefault}{pplj}
% \renewcommand{\sfdefault}{phv}

\setlength{\textwidth}{165mm} %165mm-marginparwidth
\setlength{\marginparwidth}{40mm}
\setlength{\textheight}{225mm}
\setlength{\topmargin}{-5mm}
\setlength{\oddsidemargin}{-3.5mm}
% \setlength{\parindent}{0pt}

\def\vector#1{\mbox{\boldmath $#1$}}
\newcommand{\AmSLaTeX}{%
 $\mathcal A$\lower.4ex\hbox{$\!\mathcal M\!$}$\mathcal S$-\LaTeX}
\newcommand{\PS}{{\scshape Post\-Script}}
\def\BibTeX{{\rmfamily B\kern-.05em{\scshape i\kern-.025em b}\kern-.08em
 T\kern-.1667em\lower.7ex\hbox{E}\kern-.125em X}}
\newcommand{\DeLta}{{\mit\Delta}}
\renewcommand{\d}{{\rm d}}
\def\wcaption#1{\caption[]{\parbox[t]{100mm}{#1}}}
\def\rm#1{\mathrm{#1}}
\def\tempC{^\circ \rm{C}}

\makeatletter
\def\section{\@startsection {section}{1}{\z@}{-3.5ex plus -1ex minus -.2ex}{2.3ex plus .2ex}{\Large\bf}}
\def\subsection{\@startsection {subsection}{2}{\z@}{-3.25ex plus -1ex minus -.2ex}{1.5ex plus .2ex}{\normalsize\bf}}
\def\subsubsection{\@startsection {subsubsection}{3}{\z@}{-3.25ex plus -1ex minus -.2ex}{1.5ex plus .2ex}{\small\bf}}
\makeatother

\makeatletter
\def\@seccntformat#1{\@ifundefined{#1@cntformat}%
   {\csname the#1\endcsname\quad}%      default
   {\csname #1@cntformat\endcsname}%    enable individual control
}

% proof enviroment
\renewenvironment{proof}[1][\proofname]{\par
  \pushQED{\qed}%
  \normalfont \topsep6\p@\@plus6\p@\relax
  \trivlist
  \item\relax
  {\bfseries
  #1\@addpunct{.}}\hspace\labelsep\ignorespaces
}{%
  \popQED\endtrivlist\@endpefalse
}
\makeatother

\newcommand{\tenexp}[2]{#1\times10^{#2}}


\begin{document}
% タイトル
\begin{center}
{\Large{\bf 情報工学工房第1回レポート}} \\
{\bf 電気通信大学 Ⅰ類 コンピュータサイエンスプログラム 3年} \\
{\bf 2311081 木村慎之介} \\
\end{center}

\section{はじめに}
本課題においては、プログラムを実装する言語としてpythonを用いた。

\section{ニム和とニム積を求めるプログラム}
\subsection{ニム和を求めるプログラム}
\hspace{1em}まずニム和の定義を確認する。

\begin{dfn}[ニム和]
2つの非負整数\(a, b\)に対し、その排他的論理和をニム和という
\end{dfn}

ニム和を求めるプログラムは以下のように実装した。

\begin{lstlisting}[caption={ニム和を求めるプログラム}, label=code_nim_sum]
def nim_sum(a, b):
  return a ^ b
\end{lstlisting}

ニム和の定義は排他的論理和であったから、2つの引数\(a, b\)に対して、そのXORを返す処理を行っている。
% 実行例は以下のようになった。

% \begin{lstlisting}[caption={ニム和の実行例}, label=output_nim_sum]
% >>> nim_calc.nim_sum(1, 1)
% 0
% >>> nim_calc.nim_sum(1, 0)
% 1
% >>> nim_calc.nim_sum(0, 0)
% 0
% >>> nim_calc.nim_sum(1, 3)
% 2
% \end{lstlisting}

% 以上より、ニム和が正しく実行できていることが確認できた。

\subsection{ニム積を求めるプログラム}
\hspace{1em}ニム積は最小除外数というものによって定義されていたため、まず最小除外数の定義から確認する。

\begin{dfn}[最小除外数]
\(T\)を\(\mathbb{N}_0\)の真部分集合であるとする。\(T\)の最小除外数\(\text{mex}(T)\)は次のように定義される。
\begin{equation}
\text{mex}(T) = \min(\mathbb{N}_0 \backslash T)
\end{equation}
\label{dfn_mex}
\end{dfn}

定義\ref{dfn_mex}に基づいて最小除外数を求める関数mexを以下のように実装した。

\begin{lstlisting}[caption={最小除外数を求める関数}, label=code_mex]
"""
Calculate mex value.

Values
----------
T: list
Finate subsete of N_0.

Returns
----------
i: int
Mex.
"""
i = 0

while((i in T)):
  i += 1

return i
\end{lstlisting}

次にニム積のを求めるプログラムを実装する。
ニム積は最小除外数を用いて以下のように定義される演算である。

\begin{dfn}[ニム積]
非負整数\(a, b\)が与えられた時、以下の演算をニム積と定義する。
\begin{equation}
a \otimes b = \text{mex}(\{(a' \otimes b) \oplus (a \otimes b') \oplus (a' \otimes b') | 0 \leq a' < a, 0 \leq b' < b\})
\end{equation}
\label{dfn_nim_times}
\end{dfn}

定義\ref{dfn_nim_times}をもとにニム積を求める関数nim\_timesを以下のように実装した。

\begin{lstlisting}[caption={ニム積を求める関数}, label=code_nim_times]
def nim_times(a, b):
  if(a == 0 or b == 0):
    return 0
  
  T = []
  
  for a_prime in range(0, a):
    for b_prime in range(0, b):
      t = nim_sum(nim_times(a_prime, b), nim_times(a, b_prime))
      t = nim_sum(t, nim_times(a_prime, b_prime))

      if(not(t in T)):
        T.append(t)

  return mex.mex(T)
\end{lstlisting}

コード\ref{code_nim_times}では基底として\(a = 0 \lor b = 0\)のときに\(0\)を返し、それ以外のときはコード\ref{code_nim_sum}とコード\ref{code_mex}定義したニム和と最小除外数を求める関数を用いて再帰的にニム積を求めている。
\(a = 0 \lor b = 0\)のときにニム積が\(0\)になることは\(0 \leq x < 0\)を満たす実数\(x\)が存在しないことから集合\(\{(a' \otimes b) \oplus (a \otimes b') \oplus (a' \otimes b') | 0 \leq a' < a, 0 \leq b' < b\}\)が空集合となることより求められる。\\

\section{CHOMPの先手必勝戦略}
\subsection{CHOMPの説明}
\hspace{1em}CHOMPとは、左下の1マスに毒がある板チョコを交互に食べていき、毒のマスのチョコを食べた人が負になるというゲームである。
チョコを食べるルールとしては、マスを選んだとき、そのマスの右一列と上一列を含む右上すべての領域を食べることがルールである。


\subsection{\(k \times l\)マスのCHOMPが先手必勝であることの証明}
\hspace{1em}次にCHOMPに関する以下の命題を証明する\cite{chomp_existence_theorem}。
なお、以下の証明に出てくる「先手」と「後手」は、ゲーム開始時に最初に手を打つプレイヤーを「先手」、そうでないプレイヤーを「後手」と呼ぶことに注意する。

\begin{prop}
\(k \times l(k, l > 1)\)の長方形のCHOMPの局面では、必ず先手のプレイヤーが必勝戦略を持つ。
\label{prop_chomp}
\end{prop}

\begin{proof}[命題\ref{prop_chomp}の証明]
背理法を用いて証明する。

\(k \times l(k, l > 1)\)長方形の局面において、後手が必勝戦略を持つとする。
すると、先手がどのような手をとっても必ず後手に有利な手が存在することになる。 \\
さて、ここで先手が最初に一番右上の1マスを取ったとする。
このとき後手となっていたプレイヤーは先手がマスを取った行をすべて取ることで、\((k-1) \times l\)の長方形の局面を先手に押し付けることができる。
この局面もまた長方形の局面であったから先手のプレイヤーがどのような手を打っても必ず後手が勝つ戦略が存在することになる。 \\
しかし、先手は一番上の行を取り除く手を打ち、\((k-1) \times l\)局面を後手に渡すことができてしまい、先手が必勝戦略を持つことになってしまう。
これは最初の仮定に矛盾する。
よって、先手が必勝戦略を持つ。
\end{proof}

\begin{thebibliography}{99}
\bibitem{combination_game_theory} 安福智明, 坂井公, 末續鴻輝. 組み合わせゲーム理論の世界〜数学で解き明かす必勝法〜, 共立出版株式会社, 2024.
\bibitem{chomp_existence_theorem} Doron Zeilberger. Three-Rowed CHOMP. Advances in Applied Mathematics. 2001, vol.26, no.2, p.168-179. 
\end{thebibliography}

%%%%%%%%%%%%%%%%%%%%%%%%%%%%%%%%%%%%%%%%%%%%%%%%%%%%%%%%%%%%%%%%%%%%%%
\appendix
\setcounter{figure}{0}
\setcounter{table}{0}
\numberwithin{equation}{section}
\renewcommand{\thetable}{\Alph{section}\arabic{table}}
\renewcommand{\thefigure}{\Alph{section}\arabic{figure}}
%\def\thesection{付録\Alph{section}}
\makeatletter 
\newcommand{\section@cntformat}{付録 \thesection:\ }
\makeatother
%%%%%%%%%%%%%%%%%%%%%%%%%%%%%%%%%%%%%%%%%%%%%%%%%%%%%%%%%%%%%%%%%%%%%%

    
\end{document}