\documentclass[uplatex,dvipdfmx,a4paper,10pt]{jsarticle}
\usepackage{graphicx}
\usepackage{amsmath}
\usepackage{latexsym}
\usepackage{multirow}
\usepackage{url}
\usepackage[separate-uncertainty]{siunitx}
\usepackage{physics}
\usepackage{enumerate}
\usepackage{bm}
\usepackage{pdfpages}
\usepackage{pxchfon}
\usepackage{tikz}
\usepackage{float}
\usepackage{listings}
\usepackage{amssymb}

% lstlistingのsetting
\lstset{
    basicstyle={\ttfamily},
    identifierstyle={\small},
    commentstyle={\smallitshape},
    keywordstyle={\small\bfseries},
    ndkeywordstyle={\small},
    stringstyle={\small\ttfamily},
    frame={tb},
    breaklines=true,
    columns=[l]{fullflexible},
    numbers=left,
    xrightmargin=0zw,
    xleftmargin=3zw,
    numberstyle={\scriptsize},
    stepnumber=1,
    numbersep=1zw,
    lineskip=-0.5ex
}

% tikz setting
\usepackage{tikz}
\usetikzlibrary{automata, intersections, calc, arrows, positioning, arrows.meta}

% theories setting (for japanese language)
\usepackage{amsmath}
\usepackage{amsthm}

\theoremstyle{definition}
\newtheorem{thm}{定理}[section]
\newtheorem{lem}[thm]{補題}
\newtheorem{prop}[thm]{命題}
\newtheorem{cor}[thm]{系}
\newtheorem{ass}[thm]{仮定}
\newtheorem{conj}[thm]{予想}
\newtheorem{dfn}[thm]{定義}
\newtheorem{rem}[thm]{注}

\newtheorem*{thm*}{定理}
\newtheorem*{lem*}{補題}
\newtheorem*{prop*}{命題}
\newtheorem*{cor*}{系}
\newtheorem*{ass*}{仮定}
\newtheorem*{conj*}{予想}
\newtheorem*{dfn*}{定義}
\newtheorem*{rem*}{注}

% \renewcommand{\rmdefault}{pplj}
% \renewcommand{\sfdefault}{phv}

\setlength{\textwidth}{165mm} %165mm-marginparwidth
\setlength{\marginparwidth}{40mm}
\setlength{\textheight}{225mm}
\setlength{\topmargin}{-5mm}
\setlength{\oddsidemargin}{-3.5mm}
% \setlength{\parindent}{0pt}

\def\vector#1{\mbox{\boldmath $#1$}}
\newcommand{\AmSLaTeX}{%
 $\mathcal A$\lower.4ex\hbox{$\!\mathcal M\!$}$\mathcal S$-\LaTeX}
\newcommand{\PS}{{\scshape Post\-Script}}
\def\BibTeX{{\rmfamily B\kern-.05em{\scshape i\kern-.025em b}\kern-.08em
 T\kern-.1667em\lower.7ex\hbox{E}\kern-.125em X}}
\newcommand{\DeLta}{{\mit\Delta}}
\renewcommand{\d}{{\rm d}}
\def\wcaption#1{\caption[]{\parbox[t]{100mm}{#1}}}
\def\rm#1{\mathrm{#1}}
\def\tempC{^\circ \rm{C}}

\makeatletter
\def\section{\@startsection {section}{1}{\z@}{-3.5ex plus -1ex minus -.2ex}{2.3ex plus .2ex}{\Large\bf}}
\def\subsection{\@startsection {subsection}{2}{\z@}{-3.25ex plus -1ex minus -.2ex}{1.5ex plus .2ex}{\normalsize\bf}}
\def\subsubsection{\@startsection {subsubsection}{3}{\z@}{-3.25ex plus -1ex minus -.2ex}{1.5ex plus .2ex}{\small\bf}}
\makeatother

\makeatletter
\def\@seccntformat#1{\@ifundefined{#1@cntformat}%
   {\csname the#1\endcsname\quad}%      default
   {\csname #1@cntformat\endcsname}%    enable individual control
}

% proof enviroment
\renewenvironment{proof}[1][\proofname]{\par
  \pushQED{\qed}%
  \normalfont \topsep6\p@\@plus6\p@\relax
  \trivlist
  \item\relax
  {\bfseries
  #1\@addpunct{.}}\hspace\labelsep\ignorespaces
}{%
  \popQED\endtrivlist\@endpefalse
}
\makeatother

\newcommand{\tenexp}[2]{#1\times10^{#2}}


\begin{document}
% タイトル
\begin{center}
{\Large{\bf 情報工学工房第3回レポート}} \\
{\bf 電気通信大学 Ⅰ類 コンピュータサイエンスプログラム 3年} \\
{\bf 2311081 木村慎之介} \\
\end{center}

\section{Wythoffニムのグランディ数}
\hspace{1em}本レポートではWythoffニムのグランディ数について、調べたことを報告する。

\subsection{Wythoffニムのグランディ数を求めるアルゴリズム}
\hspace{1em}まず、Wythoffニムのグランディ数を求めるアルゴリズムについて説明する。
なお、アルゴリズムの都合上\(n \times n\)のボード上すべてのグランディ数を求めることを前提とする。

\begin{enumerate}
  \item \((0, 0)\)のグランディ数を求める(これはグランディ数の定義から0となる)
  \item \(k = 1, 2, \cdots , 2n + 1\)としてステップを踏みながら順番に、\(i + j = k\)となる盤面\((i, j)\)について縦、横、斜めとグランディ数を探索しながら盤面\((i, j)\)のグランディ数を求める
\end{enumerate}

このアルゴリズムによって\(n \times n\)のWythoffニムのすべての盤面のグランディ数を正しく求まることを証明する。

\begin{proof}[アルゴリズムの正当性の証明] \\
 \hspace{1em}\(k = 0\)の時、\(i + j = 0\)を満たす盤面は\((0, 0)\)のみであり、これはアルゴリズムの説明でも述べたとおり\(0\)と求まる。\\
 \hspace{1em}\(k = l\)の時、\(i + j \leq l\)を満たす盤面\((i, j)\)のすべてのグランディ数が正しく求まっていると仮定して、\(k = l + 1\)の場合を証明する。
 \(i + j = l + 1\)を満たす任意の盤面のグランディ数は 

 \begin{equation}
 \mathcal{G}((i, j)) = \text{mex}(\{\mathcal{G}((i - m, j))|1 \leq m \leq i\} \cup \{\mathcal{G}((i, j - m)) | 1 \leq m \leq j\} \cup \{\mathcal{G}(i - m, j - m) |\ 1 \leq m \leq \min(i, j)\}) 
 \end{equation}
 
 \noindent と求まる。ここで、mexの中の各集合は順番に盤面を上に探索した時、左の探索した時、左斜め上に探索したときに見つかったグランディ数である。
 この操作で、盤面\((i, j)\)から1手で遷移できる盤面をすべて探索したことになり、かつ仮定より盤面の軸の和が\(l\)以下のグランディ数は全て求まっている。
 さらに、盤面\((i - m, j)\)、\((i, j - m)\)、\((i - m, j - m)\)の和はすべて\(l\)以下となっている。 \\
 \hspace{1em}以上より、説明したアルゴリズムで正しく盤面のグランディ数が求まる。
\end{proof}

次に、以上のアルゴリズムを実装したpythonのコードを示す。

\begin{lstlisting}[caption={Wythoffニムのグランディ数を求めるプログラム}, label={code_wythoff}]
def calculate_grundy(n):
    """
    Calculate the Grundy numbers for a Wythoff game board of size n x n.
    
    Parameters
    ----------

    n : int
        The size of the board (n x n).

    Returns
    -------
    board: list
        A 2D list representing the Grundy numbers for each position on the board.
    """
    board = [[-1 for _ in range(n + 1)] for _ in range(n + 1)]

    # calculate the Grundy number for P positions
    for i in range(n + 1):
        for j in range(n + 1):
            if is_p_position.is_p_position(i, j):
                board[i][j] = 0
            else:
                board[i][j] = -1

    # calculate the Grundy numbers for non-P positions
    for i in range(1, 2 * n + 1):
        if i <= n: # proceed for the rows
            for j in range(i + 1):
                if board[i - j][j] == -1:
                    grundy_set = set()
                    # search left, up, and leftUp for Grundy numbers
                    for k in range(1, i - j + 1):
                        # up
                        grundy_set.add(board[i - j - k][j])
                    for k in range(j + 1):
                        # left
                        grundy_set.add(board[i - j][k])
                    for k in range(1, min(i - j, j) + 1):
                        # leftUp
                        grundy_set.add(board[i - j - k][j - k])

                    # calculate the minimum excludant (mex)
                    mex = 0
                    while mex in grundy_set:
                        mex += 1

                    board[i - j][j] = mex # set the Grundy number

        else: # proceed for the columns
            for j in range(2 * n - i + 1):
                grundy_set = set()
                # search left, up, and leftUp for Grundy numbers
                if board[n - j][i - n + j] == -1:
                    for k in range(1, n - j + 1):
                        # up
                        grundy_set.add(board[n - j - k][i - n + j])
                    for k in range(1, i - n + j + 1):
                        # left
                        grundy_set.add(board[n - j][i - n + j - k])
                    for k in range(1, min(n - j, i - n + j) + 1):
                        # leftUp
                        grundy_set.add(board[n - j - k][i - n + j - k])

                    # calculate the minimum excludant (mex)
                    mex = 0
                    while mex in grundy_set:
                        mex += 1

                    board[n - j][i - n + j] = mex  # set the Grundy number

    return board
\end{lstlisting}

\begin{thebibliography}{99}
  \bibitem{combination_game_theory} 安福智明, 坂井公, 末續鴻輝. 組み合わせゲーム理論の世界〜数学で解き明かす必勝法〜, 共立出版株式会社, 2024.
\end{thebibliography}

%%%%%%%%%%%%%%%%%%%%%%%%%%%%%%%%%%%%%%%%%%%%%%%%%%%%%%%%%%%%%%%%%%%%%%
\appendix
\setcounter{figure}{0}
\setcounter{table}{0}
\numberwithin{equation}{section}
\renewcommand{\thetable}{\Alph{section}\arabic{table}}
\renewcommand{\thefigure}{\Alph{section}\arabic{figure}}
%\def\thesection{付録\Alph{section}}
\makeatletter 
\newcommand{\section@cntformat}{付録 \thesection:\ }
\makeatother
%%%%%%%%%%%%%%%%%%%%%%%%%%%%%%%%%%%%%%%%%%%%%%%%%%%%%%%%%%%%%%%%%%%%%%

    
\end{document}